\begin{comment}
Follow this guide, step by step

\subsubsection*{Requirements}
\begin{itemize}
    \item Linux cluster (Raven, Cobra)
    \item R 4.0.2
\end{itemize}

\subsubsection*{Process}

Paste the following in the terminal (**change `username` with your user name**):

Connect to the cluster computer:

\begin{verbatim}
ssh username@raven.mpcdf.mpg.de
\end{verbatim}

\subsubsection*{R packages installation}

Load interpreters

\begin{verbatim}
cd ~
module purge
module load jdk/8.265 gcc/10 impi/2021.2 fftw-mpi R/4.0.2
\end{verbatim}

Install libtiff

\begin{verbatim}
# Install libtiff
mkdir libtiff
cd libtiff
wget https://download.osgeo.org/libtiff/tiff-4.3.0.tar.gz
tar -xvf tiff-4.3.0.tar.gz
mkdir install
cd tiff-4.3.0
mkdir compile
cd compile
# Change username here
../configure --prefix=/u/username/libtiff/install
make
make install
# Change username here
export PKG\_CONFIG\_PATH=/u/username/libtiff/install/lib/pkgconfig/
\end{verbatim}

Change directory to home. Type `~` and press [Enter]

Load R. Enter `R` on the terminal and press [Enter]

Paste the following in the R console inside the terminal:

\begin{verbatim}
if("pacman" %in% rownames(installed.packages()) == FALSE)
{install.packages("pacman")}
\end{verbatim}

When prompted, type `yes` to install and `yes` to create a personal library. After, another prompt will appear. Select which repository you would like to use. Enter `1` (cloud) and then press [Enter]

Then, paste the following:

\begin{verbatim}
pacman::p\_load(ijtiff)
\end{verbatim}


After, paste the following:

\begin{verbatim}
pacman::p\_load(XML)
\end{verbatim}

Lastly, paste the following:

\begin{verbatim}
pacman::p\_load(dplyr, stringr, parallel, tidyr, data.table, ff, dtplyr, compiler, changepoint, R.utils, lemon, ggquiver, ggplot2, ggdark, scales, ggforce, viridis, RcppRoll, metR)
\end{verbatim}

Exit R by typing `q()` and then `N` to not save

\subsubsection*{Python packages installation}

Create Python Packages list. Open the terminal text editor, then type `nano` and paste:

\begin{verbatim}
aiohttp
aiohttp-cors
aioredis
appdirs
async-timeout
attrs
blessings
boto3
botocore
cachetools
certifi
chardet
click
colorama
colorful
cycler
decorator
distlib
et-xmlfile
filelock
future
google-api-core
google-auth
googleapis-common-protos
gpustat
grpcio
hiredis
idna
imageio
imglyb
jgo
jmespath
JPype1
jsonschema
kiwisolver
matplotlib
msgpack
multidict
nd2reader
networkx
numpy
nvidia-ml-py3
opencensus
opencensus-context
opencv-python
openpyxl
packaging
pandas
Pillow
PIMS
pims-nd2
pipenv
prometheus-client
protobuf
psutil
py-spy
pyasn1
pyasn1-modules
pyparsing
pyrsistent
python-dateutil
pytz
PyWavelets
PyYAML
ray
requests
rsa
s3transfer
scikit-image
scipy
scyjava
six
slicerator
tifffile
typing-extensions
urllib3
virtualenv
virtualenv-clone
xarray
xlrd
xmltodict
yarl
\end{verbatim}

Close nano:
\begin{itemize}
    \item Press [CTRL] + [X] to close
    \item Press [Y] to save
    \item Save as `python\_requirements.txt`
\end{itemize}

> **Optional**: You may use screen to let process run on the background
>    * Type `screen`
>    * Wait 5s to load
>    * Press [CTRL] + [A] and let go
>    * Press [D]
>    * To resume, type `screen -r`
>> \_If there's more than one screen, type `screen -r` . to get the index of screens and then replace 00000 with the index (`screen -r 00000`)\_

\begin{verbatim}
# Install ImageJ
wget https://downloads.imagej.net/fiji/latest/fiji-linux64.zip
unzip fiji-linux64.zip

# Install conda
wget https://repo.anaconda.com/miniconda/Miniconda3-latest-Linux-x86\_64.sh
chmod +x Miniconda3-latest-Linux-x86\_64.sh
./Miniconda3-latest-Linux-x86\_64.sh
export PATH=~/miniconda/bin:$PATH
source ~/miniconda3/bin/activate
export PATH="/miniconda3/bin":$PATH

# Follow Terminal instructions to install conda

# Create conda environment
conda create -n dynamics\_pipeline python=3.8 anaconda
\end{verbatim}

Then paste,

\begin{verbatim}
# Create conda environment
conda activate dynamics\_pipeline

# Install conda packages
conda install -c conda-forge setuptools
conda install -c conda-forge javabridge
conda install -c conda-forge libxml2
pip install glib

# Install Python packages
python -m pip install --user -r python\_requirements.txt
\end{verbatim}

\subsubsection*{Git installation}

Paste the following and rename the last element to yours:

\begin{verbatim}
git config --global user.name "username"
git config --global user.email email@mpcdf.mpg.de
ssh-keygen -t rsa -b 4096 -C "name@raven.mpcdf.mpg.de"
\end{verbatim}

Press [Enter] to skip some steps and get the ssh key. Then, copy the entire block of string from start to finish, including the \_ssh-rsa\_ and the ending name . Use `cat /u/username/.ssh/id\_rsa.pub` and change **user\_path** accordingly

Sign-in to GitHub, https://github.com/login

Paste the key into GitHub, https://github.com/settings/keys

\begin{itemize}
    \item New SSH key
    \item Type in the cluster name as **Title** and paste the key under **Key**
\end{itemize}

Create a directory to save the pipeline scripts

\begin{verbatim}
mkdir dynamics\_pipeline
cd dynamics\_pipeline
\end{verbatim}

Paste the following in the terminal to clone the pipeline:

\begin{verbatim}
git init
git remote add origin git@github.com:MJ-Taylor-Lab/DynamicsPipeline.git
git remote set-url origin git@github.com:MJ-Taylor-Lab/DynamicsPipeline.git
git fetch --all
git pull origin master
\end{verbatim}

---
\subsection*{Image Analysis Pipeline}
\end{comment}

\subsubsection*{Input}
The input data goes into ~/new\_pipeline/pending\_processing/batch\_date/Input/

\begin{table}[htb]
  \centering
  \caption{Numbers which will be constant throughout the analysis}
  \label{tab:constants}
  \begin{tabular}{ | l | c | p{5cm} | }
    \hline
    \textbf{Parameter} & \textbf{Value} & \textbf{Comments} \\ \hline
    \texttt{tiff\_compression\_level} & 5 & out of 10 \\ \hline
    \texttt{cell\_diameter} & 25 & px, odd number \\ \hline
    \texttt{puncta\_diameter} & 5 & px, odd number \\ \hline
  \end{tabular}
\end{table}

\begin{table}[htb]
  \centering
  \caption{The dark frame is the camera noise. This typically is 1000 frames averaged, though 50 frames could do, so long as the standard deviation does not change with more images added. It should be at the same exposure as the images using the same camera as the microscopy images. Thus, one image could be used for multiple channels.}
  \label{tab:dark\_frames}
  \begin{tabular}{ | l | c | }
    \hline
    \textbf{Image} & \textbf{Exposure} \\ \hline
    \texttt{20201026 Darkfield 200ms binned.tif} & 200 ms \\ \hline
    \texttt{20201026 Darkfield 50ms binned.tif} & 50 ms \\ \hline
    \texttt{20201026 Darkfield 100ms binned.tif} & 100 ms \\ \hline
  \end{tabular}
\end{table}

\begin{table}[htb]
  \centering
  \caption{Directory paths}
  \label{tab:directories}
  \begin{tabular}{ | l | p{10cm} | }
    \hline
    \textbf{Contains} & \textbf{Path} \\ \hline
    \texttt{input} & \texttt{\textasciitilde/Input} \\ \hline
    \texttt{processing} & \texttt{\textasciitilde/Processing} \\ \hline
    \texttt{output} & \texttt{\textasciitilde/Output} \\ \hline
    \texttt{dark\_frames} & \texttt{\textasciitilde/dark\_frames} \\ \hline
    \texttt{flat\_fields} & \texttt{\textasciitilde/flat\_fields} \\ \hline
    \texttt{ImageJ} & \texttt{\textasciitilde/Fiji.app/ImageJ-linux64} \\ \hline
  \end{tabular}
\end{table}


\subsubsection*{exclusion\_channels.csv}
\begin{tabular}{|c|}
    \hline
    value \\
    \hline
    IL-1 \\
    Brightfield \\
    WideField \\
    \hline
\end{tabular}

\subsubsection*{images.csv}
\begin{tabular}{|l|l|l|l|l|l|l|l|l|l|l|l|}
    \hline
    image & cohort & segment\_with & ligand & ligand\_density & trackmate\_max\_link\_distance & trackmate\_threshold & trackmate\_frame\_gap & T Cy5 protein\_name & T GFP protein\_name & T RFP protein\_name & WideField protein\_name \\
    \hline
    20211218 0p8nM 069-1R\_TRAF6\_MyD88 Grid\_1um\_11mol 001.nd2 & MyD88 TRAF6 1um\_grid & MyD88 & 0.8 nM IL-1 & 11 & 5 & 1.5 & 5 & IL-1 & MyD88 & TRAF6 & Brightfield \\
    20211218 GFP calibration\_10pct\_60ms 005.nd2 & Calibrations & GFP &  &  & 2.5 & 1.5 & 5 & IL-1 & GFP & mScarlet & Brightfield \\
    20211218 mScarlet calibration\_10pct\_60ms 001.nd2 & Calibrations & mScarlet &  &  & 2.5 & 1.5 & 5 & IL-1 & GFP & mScarlet & Brightfield \\
    \hline
\end{tabular}

\begin{comment}
\subsection*{Run}

Connect to the cluster computer:

\begin{verbatim}[language=bash]
ssh username@raven.mpcdf.mpg.de
\end{verbatim}

If you need the latest scripts, paste in the Terminal:

\begin{verbatim}[language=bash]
cd dynamics\_pipeline
git pull origin master
\end{verbatim}

\subsubsection*{SLURM Instructions}

Pull the scripts before using \lstinline{git pull origin master} and modify the parameters of \lstinline{submit\_node.sh} accordingly.

\begin{verbatim}
#!/bin/bash -l
    
#SBATCH -o ./job.out.%j
#SBATCH -e ./job.err.%j
#SBATCH -D ./
#SBATCH -J 20211218
#SBATCH --mail-type=ALL
#SBATCH --mail-user=email@mpcdf.mpg.de
#SBATCH --nodes=1
#SBATCH --ntasks-per-node=1
#SBATCH --cpus-per-task=72
#SBATCH --time=24:00:00
    
# Load all needed packages
module purge
module load jdk/8.265 gcc/10 impi/2021.2 fftw-mpi R/4.0.2
echo 'modules loaded'
conda activate dynamics\_pipeline
echo 'conda activated'
    
# Specify parameters
## Path of parameters table
## Change username to your cluster user name
path=$'/raven/u/username/new\_pipeline/pending\_processing/batch\_date/Input/parameter\_tables'
    
## Scripts folder
## Change username to your cluster user name
cd /raven/u/username/dynamics\_pipeline
    
## Cores for parallel processing in R
export OMP\_NUM\_THREDS=144
    
# Run scripts
## Python scripts
python mission\_control.py $path 12
    
## Run R Scripts
Rscript --vanilla --verbose r\_scripts/extract\_intensity.R $path
Rscript --vanilla --verbose r\_scripts/colocalization.R $path
Rscript --vanilla --verbose r\_scripts/compile\_tables.R $path
#Rscript --vanilla --verbose r\_scripts/compress\_everything.R $path
    
sleep 10
\end{verbatim}

Paste `sbatch submit\_node.sh` to submit to SLURM.

\subsection*{Output}

\end{comment}
\subsubsection*{Essentials.csv.gz}

\textbf{Identification}

\begin{itemize}
    \item RELATIVE\_PATH: Relative path to cell folder. Simplifies address to source images and parameters
    \item COHORT: Cell line name (proteins tagged) plus any perturbations (for example, grids, inhibitors)
    \item IMAGE: Name of image. Our format is:
    \begin{itemize}
        \item Date (YYYYMMDD)
        \item Ligand concentration + density
        \item Cell line name
        \item Plate + well number
    \end{itemize}
    \item PROTEIN: Protein name
    \item UNIVERSAL\_TRACK\_ID: Unique cluster identifier, computed as:
    \begin{itemize}
        \item IMAGE + '...'
        \item CELL + '...'
        \item PROTEIN + '...'
        \item TRACK\_ID
    \end{itemize}
    \item UNIVERSAL\_SPOT\_ID: Unique spot identifier, computed as:
    \begin{itemize}
        \item UNIVERSAL\_TRACK\_ID + '...'
        \item FRAME
    \end{itemize}
    \item ANALYSIS\_TIME\_STAMP: Date and time of analysis completion
\end{itemize}

\subsubsection*{Temporal measurements}

\begin{itemize}
    \item TIME: Time in seconds from when image acquisition started
    \item FRAME: Image frame number
    \item TIME\_SINCE\_LANDING: Time in seconds since the first spot in the cell appeared
    \item FRAMES\_SINCE\_LANDING: Frames since the first spot in the cell appeared
    \item TIME\_ADJUSTED: Cluster time in seconds
    \item FRAMES\_ADJUSTED: Cluster time in frames
    \item LIFETIME: Cluster time in seconds. May need to be recalculated after passing fi
\end{itemize}

I recommend calculating the fluorophore bleaching rate. Filter data (FRAMES\_SINCE\_LANDING, FRAMES\_ADJUSTED) based on the results of this parameter.

\subsubsection*{Spatial measurements}

\begin{itemize}
    \item ABSOLUTE\_POSITION\_X: X-coordinate of cluster centroid in microns
    \item ABSOLUTE\_POSITION\_Y: Y-coordinate of cluster centroid in microns
    \item CELL\_AREA: Area of the cell in microns
    \item NEAREST\_SPOT: Distance to nearest cluster in pixels
    \item SPOTS\_WITHIN\_RADIUS: Number of spots within puncta radius
\end{itemize}

\subsubsection*{Amount of substance data}

\begin{itemize}
    \item NORMALIZED\_INTENSITY: Estimate number of molecules of the reference protein
    \item STARTING\_NORMALIZED\_INTENSITY: Starting amount of the reference protein
    \item MAX\_NORMALIZED\_INTENSITY: Max amount (brightness) of the relative protein
    \item START\_TO\_MAX\_INTENSITY: Growth, measured as max – start amount
    \item COMPLEMENTARY\_PROTEIN\_\#: Protein in other channel(s)
    \item COMPLEMENTARY\_TOTAL\_INTENSITY\_\#: Brigness of other channel in arbitrary units
    \item COMPLEMENTARY\_NORMALIZED\_INTENSITY\_\#: Estimate number of molecules of the query protein
    \item COMPLEMENTARY\_UNIVERSAL\_SPOT\_ID\_\#: UNIVERSAL\_TRACK\_ID of the query protein spot
\end{itemize}


\subsubsection*{Parameters.csv.gz}

\subsubsection*{Other information}
\begin{itemize}
    \item \texttt{RELATIVE\_PATH}: Identifies cell + protein in question
    \item \texttt{LIGAND}: Ligand that stimulates 
    \item \texttt{SEGMENT\_WITH}: Protein name of the channel that was used for segmenting the cells from the image
\end{itemize}

\subsubsection*{Fluorophore data}
\begin{itemize}
    \item \texttt{CALIBRATION\_IMAGE}: Image used for fluorophore normalization
    \item \texttt{CALIBRATION\_TOTAL\_INTENSITY}: Median brightness of the fluorophore in arbitrary units
    \item \texttt{CALIBRATION\_STANDARD\_DEVIATION}: Variance of the brightness of the fluorophore in arbitrary units
\end{itemize}

\subsubsection*{Microscope information}
\begin{itemize}
    \item \texttt{CHANNEL}: Microscope channel
    \item \texttt{POWER}: Laser power
    \item \texttt{EXCITATION}: Peak wavelength of laser excitation
    \item \texttt{EMMISION}: Peak wavelength of emmision filter
    \item \texttt{ANGLE}: TIRF critical angle in degrees
    \item \texttt{DIRECTION}: Refraction direction in degrees (angle)
    \item \texttt{FOCUS}: Objective z-axis distance (not the stage z-axis)
    \item \texttt{OBJECTIVE}: Objective magnifying power
    \item \texttt{TIME\_START}: Timestamp of when imaging acquisition started
    \item \texttt{FRAME\_RATE}: Number of frames per second (Hz)
\end{itemize}

\subsubsection*{Spatial information}
\begin{itemize}
    \item \texttt{WIDTH}: Image width in microns
    \item \texttt{HEIGHT}: Image height in microns
    \item \texttt{CALIBRATION\_UM}: Pixel size in microns
    \item \texttt{CELL\_DIAMETER}: Estimate cell diameter, as entered in pipeline. Used in the cell median-filter step, whose resulting image is \texttt{PROTEIN + '\_intensity\_ref.tif'}
    \item \texttt{PUNCTA\_DIAMETER}: Estimate puncta diameter, as entered in pipeline. Used in the puncta median-filter step, whose resulting image is \texttt{PROTEIN + '\_tracking\_ref.tif'}
    \item \texttt{SPOT\_RADIUS\_LIMIT}: Radius of spot
    \item \texttt{CELL\_POSITION\_X}: X-coordinate of the cell in the image
    \item \texttt{CELL\_POSITION\_Y}: Y-coordinate of the cell in the image
\end{itemize}

\subsubsection*{TrackMate information}
\begin{itemize}
  \item \verb|TRACKMATE_THRESHOLD|: TrackMate's threshold
  \item \verb|TRACKMATE_FRAME_GAP|: TrackMate's maximum frame gap between spots appearing at a location (missed detection)
  \item \verb|TRACKMATE_GAP_LINK_DISTANCE|: TrackMate's maximum frame gap distance in pixels between spots appearing at a location (missed detection)
  \item \verb|TRACKMATE_MAX_LINK_DISTANCE|: Maximum distance in pixels before the spot gets classified as a new distinct track (cluster)
\end{itemize}
